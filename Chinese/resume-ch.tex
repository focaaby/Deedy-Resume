%%%%%%%%%%%%%%%%%%%%%%%%%%%%%%%%%%%%%%%
% Deedy - One Page Two Column Resume
% LaTeX Template
% Version 1.2 (16/9/2014)
%
% Original author:
% Debarghya Das (http://debarghyadas.com)
%
% Original repository:
% https://github.com/deedydas/Deedy-Resume
%
% IMPORTANT: THIS TEMPLATE NEEDS TO BE COMPILED WITH XeLaTeX
%
% This template uses several fonts not included with Windows/Linux by
% default. If you get compilation errors saying a font is missing, find the line
% on which the font is used and either change it to a font included with your
% operating system or comment the line out to use the default font.
%
%%%%%%%%%%%%%%%%%%%%%%%%%%%%%%%%%%%%%%
%
% TODO:
% 1. Integrate biber/bibtex for article citation under publications.
% 2. Figure out a smoother way for the document to flow onto the next page.
% 3. Add styling information for a "Projects/Hacks" section.
% 4. Add location/address information
% 5. Merge OpenFont and MacFonts as a single sty with options.
%
%%%%%%%%%%%%%%%%%%%%%%%%%%%%%%%%%%%%%%
%
% CHANGELOG:
% v1.1:
% 1. Fixed several compilation bugs with \renewcommand
% 2. Got Open-source fonts (Windows/Linux support)
% 3. Added Last Updated
% 4. Move Title styling into .sty
% 5. Commented .sty file.
%
%%%%%%%%%%%%%%%%%%%%%%%%%%%%%%%%%%%%%%%
%
% Known Issues:
% 1. Overflows onto second page if any column's contents are more than the
% vertical limit
% 2. Hacky space on the first bullet point on the second column.
%
%%%%%%%%%%%%%%%%%%%%%%%%%%%%%%%%%%%%%%


\documentclass[]{deedy-resume-openfont-ch}
\usepackage{fancyhdr}

\pagestyle{fancy}
\fancyhf{}

\begin{document}

%%%%%%%%%%%%%%%%%%%%%%%%%%%%%%%%%%%%%%
%
%     LAST UPDATED DATE
%
%%%%%%%%%%%%%%%%%%%%%%%%%%%%%%%%%%%%%%
\lastupdated

%%%%%%%%%%%%%%%%%%%%%%%%%%%%%%%%%%%%%%
%
%     TITLE NAME
%
%%%%%%%%%%%%%%%%%%%%%%%%%%%%%%%%%%%%%%
\namesection{王茂林}{ \urlstyle{same}\href{https://focaaby.github.io/}{https://focaaby.github.io}\\
0926-633005 | \href{mailto:focaaby@gmail.com}{focaaby@gmail.com}
}

%%%%%%%%%%%%%%%%%%%%%%%%%%%%%%%%%%%%%%
%
%     COLUMN ONE
%
%%%%%%%%%%%%%%%%%%%%%%%%%%%%%%%%%%%%%%

\begin{minipage}[t]{0.45\textwidth}


%%%%%%%%%%%%%%%%%%%%%%%%%%%%%%%%%%%%%%
%     EXPERIENCE
%%%%%%%%%%%%%%%%%%%%%%%%%%%%%%%%%%%%%%

\section{經歷}

\runsubsection{富邦金控創新科技辦公室}
\descript{| 悍將計畫實習生}
\location{Sep 2017 – Jan 2018 | Taipei, Taiwan}
\vspace{\topsep} % Hacky fix for awkward extra vertical space
\begin{tightemize}
    \item 使用 Ethereum 及 web3j 開發區塊鏈應用程式
\end{tightemize}
\sectionsep

\runsubsection{軟體定義資訊架構下提升應用系統穩定及安全之研究}
\descript{| 臺灣證券交易所}
\location{May 2017 – Dec 2017 | Taipei, Taiwan}
% \vspace{\topsep} % Hacky fix for awkward extra vertical space
\begin{tightemize}
    \item 重新設計架構基於 Docker 容器,並避免於單一脆弱點(SPOF)問題及高可行性(High Availability)
    \item 設計一個 CI/CD 流程及透過 Ansible 腳本自動測試環境變數
    \item 基於 Kafka 及 Zookeeper 叢集建置一個高吞吐量架構
\end{tightemize}

\runsubsection{物聯網 2.0 智慧應用服務平台}
\descript{| 正崴精密工業}
\location{Nov 2016 – Aug 2017 | Taipei, Taiwan}
% \vspace{\topsep} % Hacky fix for awkward extra vertical space
\begin{tightemize}
    \item 協助建制 IFTTT(if this, then that)流程
    \item 架設 openHAB 2 環境並測試介接 MQTT broker 程式及 AWS IoT 平台
\end{tightemize}

\runsubsection{區塊鏈投票系統}
\descript{| 前後端 + 智能合約}
\location{Feb 2017 – Jun 2017 | Taipei, Taiwan}
\begin{tightemize}
    \item 使用 Docker 建制 Ethereum 多節點及應用程式之環境
    \item 整合並撰寫前後端系統(Vue.js 及 Node.js)
\end{tightemize}

\runsubsection{意門科技}
\descript{| 前端實習生 }
\location{Jul 2015 – Jun 2016 | Puli, Taiwan}
\begin{tightemize}
    \item 監視系統整合平台:以網頁管理各家監視設備 (IPCAM \& DVR)
    \item 前期以 jQuery 處理前端技術並採用前後端分離架構
    \item 後期使用 React.js 框架重構建置 SPA
\end{tightemize}

\runsubsection{暨南大學資訊管理學系}
\descript{| 網頁伺服器維護管理員 }
\location{June 2014 – Sep 2016 | Puli, Taiwan}
% \vspace{\topsep} % Hacky fix for awkward extra vertical space
\begin{tightemize}
    \item 將 4 台實體機器移植至計算中心 VMware vSphere 環境
    \item 移植 Apache 網頁伺服器至 Nginx,並設定 HTTP/2 和 SSL,共節省了 60\% 網頁讀取速度
    \item 建置自動部署 CMS(Wordpress 或 Joomla)腳本
\end{tightemize}
\sectionsep

%%%%%%%%%%%%%%%%%%%%%%%%%%%%%%%%%%%%%%
%     PROJECT
%%%%%%%%%%%%%%%%%%%%%%%%%%%%%%%%%%%%%%
\section{個人專案}

\runsubsection{台灣電子書搜尋}
\descript{| 網頁版 }
\location{Dec 2017 – Now | Taipei, Taiwan}
\begin{tightemize}
    \item 使用 Vue.js 介接後端爬蟲搜尋跨平台電子書價錢、作者等資訊
    \item 利用 Travis CI 連動 GitHub,自動部屬至 GitHub Pages 
\end{tightemize}

\sectionsep


%%%%%%%%%%%%%%%%%%%%%%%%%%%%%%%%%%%%%%
%
%     COLUMN TWO
%
%%%%%%%%%%%%%%%%%%%%%%%%%%%%%%%%%%%%%%

\end{minipage}
\hfill
\begin{minipage}[t]{0.5\textwidth}

%%%%%%%%%%%%%%%%%%%%%%%%%%%%%%%%%%%%%%
%     EDUCATION
%%%%%%%%%%%%%%%%%%%%%%%%%%%%%%%%%%%%%%

\section{教育背景}

\runsubsection{臺灣科技大學}
\descript{| 資訊管理研究所}
\location{Sep 2016 - Aug 2018 | Taiwan}
\descript{以區塊鏈實作符合共同準則安全稽核要求的紀錄儲存系統}
\location{Keyword: Common Criteria, security auditing, blockchain, smart contract, Hypereledger}

\vspace{\topsep}
\runsubsection{國立暨南國際大學}
\descript{| 資訊管理學系}
\location{Sep 2011 - Aug 2016 | Taiwan}
\descript{Linux 核心層級入侵偵測系統反應機制的研究與實作}
\location{Keywords:Virtual Machine, Virtual Machine Monitor, Intrusion Detection and Prevention System, Linux Kernel}

\sectionsep

%%%%%%%%%%%%%%%%%%%%%%%%%%%%%%%%%%%%%%
%     AWARDS
%%%%%%%%%%%%%%%%%%%%%%%%%%%%%%%%%%%%%%

\section{履歷競賽}

\runsubsection{IOT SANDBOX 資安競賽}
\descript{| 國家高速網路與計算中心 }
\location{May 2 2018 - May 3 2018| Yilan, Taiwan}
\descript{隊名:謝嘉哥, 第一名}


\sectionsep

%%%%%%%%%%%%%%%%%%%%%%%%%%%%%%%%%%%%%%
%     SKILLS
%%%%%%%%%%%%%%%%%%%%%%%%%%%%%%%%%%%%%%

\section{專業技能}

\begin{minipage}[t]{0.45\textwidth}
    \subsection{前端技術}
    \vspace{\topsep}
    \vspace{\topsep}
    \begin{tightemize}
        \item React.js
        \item Javascript
        \item ECMAScript 6
        \item HTML/CSS/SCSS
    \end{tightemize}
\end{minipage}
\hfill
\begin{minipage}[t]{0.45\textwidth}
    \subsection{後端技術}
    \vspace{\topsep}
    \vspace{\topsep}
    \begin{tightemize}
        \item Node.js
    \end{tightemize}
\end{minipage}

\vspace{\topsep}
\begin{minipage}[t]{0.45\textwidth}
    \subsection{作業系統}
    \vspace{\topsep}
    \vspace{\topsep}
    \begin{tightemize}
        \item Ubuntu
        \item macOS
    \end{tightemize}
\end{minipage}
\hfill
\begin{minipage}[t]{0.45\textwidth}
    \subsection{Linux}
    \vspace{\topsep}
    \vspace{\topsep}
    \begin{tightemize}
        \item Shell Script
        \item Nginx
        \item Docker
        \item Ansible
    \end{tightemize}
\end{minipage}

\vspace{\topsep}
\begin{minipage}[t]{0.45\textwidth}
    \subsection{版本控制}
    \vspace{\topsep}
    \vspace{\topsep}
    \begin{tightemize}
        \item Git
        \item GitHub
        \item GitLab
    \end{tightemize}
\end{minipage}
\hfill
\begin{minipage}[t]{0.45\textwidth}
    \subsection{文書能力}
    \vspace{\topsep}
    \vspace{\topsep}
    \begin{tightemize}
        \item Word
        \item Excel
        \item LaTeX\
    \end{tightemize}
\end{minipage}

\end{minipage}
\end{document}  \documentclass[]{article}
